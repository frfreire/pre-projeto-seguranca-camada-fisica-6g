%%%%%classe do documento%%%%%
\documentclass[article,12pt, a4paper]{abntex2}
%%%%%%%%%%%%%%%%%%%%%%%%%%%%%

%%%%%%%Pacotes%%%%%%%%%%%%%%
\usepackage[utf8]{inputenc}
\usepackage[brazil]{babel}
\usepackage{graphicx}
\usepackage{booktabs}
\usepackage{xcolor}
\usepackage{float}
\usepackage[num]{abntex2cite}
\citebrackets[]
\usepackage{capa} % Carrega o capa.sty modificado
\usepackage{indentfirst}
\usepackage{cmap}
\usepackage{amsmath,amsfonts,amssymb}
% \usepackage{booktabs} % Já carregado acima
\usepackage{multirow}
\usepackage{geometry}
\usepackage{tikz}
\usepackage{pgfplots}
\usepackage{setspace}
\usepackage{fancyhdr}
\usepackage{titlesec}
%\usepackage{caption}
\usepackage{subcaption}
\usepackage{array}
\usepackage{url}
\usepackage[T1]{fontenc}
\usepackage{tabularx} 
\usepackage{enumitem} 

% Configuração de margens - MAIS ESTREITAS
\geometry{
    a4paper,
    left=2.2cm,
    right=2.2cm,
    top=2.0cm,
    bottom=2.0cm
}

% Configurações de espaçamento - REDUZIDO
\linespread{0.98}
\setlength{\baselineskip}{0.95\baselineskip}
\setlength{\parindent}{1cm}
\setlength{\parskip}{0pt}

% Reduzir espaçamentos entre seções - MAIS COMPACTO
\titleformat{\section}[block]{\normalfont\Large\bfseries}{\thesection}{0.8em}{}
\titlespacing*{\section}{0pt}{8pt plus 2pt minus 1pt}{4pt plus 1pt minus 1pt}

\titleformat{\subsection}[block]{\normalfont\large\bfseries}{\thesubsection}{0.8em}{}
\titlespacing*{\subsection}{0pt}{6pt plus 2pt minus 1pt}{3pt plus 1pt minus 1pt}

\titleformat{\subsubsection}[block]{\normalfont\normalsize\bfseries}{\thesubsubsection}{0.8em}{}
\titlespacing*{\subsubsection}{0pt}{4pt plus 1pt minus 1pt}{2pt plus 1pt minus 1pt}

% Configurações TikZ
\usetikzlibrary{shapes,arrows,positioning,calc}

% Definir cores
\definecolor{qpskred}{RGB}{255,0,0}
\definecolor{qpskblue}{RGB}{0,0,255}
\definecolor{qpskgreen}{RGB}{0,128,0}
\definecolor{qpskorange}{RGB}{255,165,0}

% Configuração de hiperlinks
\hypersetup{
    colorlinks=true,
    linkcolor=blue,
    citecolor=blue,
    filecolor=magenta,
    urlcolor=blue
}

% Tirando impressão frente e verso
\setboolean{@twoside}{false}

% --- DADOS GERAIS (ATUALIZADOS) ---
\titulo{CARACTERIZAÇÃO DE PRIMITIVAS DE SEGURANÇA FÍSICA NO DOMÍNIO TERAHERTZ PARA REDES 6G}
\autor{FABRICIO RODRIGUES FREIRE}
\local{Brasília}
\data{2025}

% 1. Instituição atualizada para "Programa..." como no modelo [cite: 1, 2, 3]
\instituicao{Universidade de Brasília \par 
             Faculdade de Tecnologia \par 
             Programa de Pós-Graduação em Engenharia Elétrica}

% 2. Tipo de trabalho atualizado com o texto EXATO que você forneceu [cite: 7]
\preambulo{Pré-projeto apresentado ao Programa de Pós-Graduação 
em Engenharia Elétrica da Faculdade de Tecnologia (PPGEE-FT/UnB), 
como exigência parcial para preenchimento de vaga no Curso de 
Doutorado Acadêmico.}

% 3. Novos campos adicionados (baseado no modelo [cite: 8, 9] e nos seus dados)
\areadeconcentracao{Telecomunicações e Redes de Comunicação}
\linhadepesquisa{Comunicações Móveis}
% --- FIM DOS DADOS GERAIS ---


\begin{document}

\imprimircapa % Este comando irá gerar a capa no formato correto

\textual

% Incluir seções
\section{INTRODUÇÃO}

A sexta geração (6G) de redes sem fio impulsiona a exploração da faixa de Terahertz (THz) (0.1-10 THz) para alcançar taxas de transmissão de terabits por segundo (Tbps) \cite{Rappaport2019}. Contudo, essa nova fronteira de conectividade massiva expõe vulnerabilidades de segurança. A criptografia convencional, baseada em complexidade computacional, enfrenta ameaças de ataques quânticos e impõe um ônus de gerenciamento de chaves em redes de Internet das Coisas (IoT) com trilhões de dispositivos de recursos limitados \cite{Saad2021, Alagic2020}.

Neste contexto, a Segurança da Camada Física (PLS) surge como uma abordagem complementar, explorando as características intrínsecas do canal de comunicação para garantir a confidencialidade \cite{Mukherjee2014}. O canal THz possui propriedades físicas distintivas, incluindo a absorção molecular seletiva em frequência (e.g., por H₂O e O₂) e a predominância de enlaces de linha de visada (LoS) \cite{Han2018}. Essas características, vistas como desafios de propagação, são espacialmente localizadas e temporalmente dinâmicas, criando uma "impressão digital" (fingerprint) recíproca entre comunicantes legítimos \cite{Ma2018}.

A hipótese central desta pesquisa é que a alta sensibilidade do canal THz às condições atmosféricas e ao ambiente físico pode ser explorada para extrair primitivas de segurança. Técnicas de PLS tradicionais, que dependem de rico multipercurso, falham em canais THz dominados por LoS \cite{Wang2017mmWave}. Esta proposta visa suprir essa lacuna, investigando a variabilidade temporal da absorção molecular atmosférica como fonte primária de aleatoriedade para geração de chaves secretas. Embora Fang et al. \cite{Fang2022} tenham demonstrado que a absorção atmosférica pode ser explorada para limitar o alcance de sinais THz e criar zonas de segurança, nenhum trabalho estabeleceu protocolo de SKG que explore as flutuações atmosféricas como fonte de entropia.

\subsection{Problema e Perguntas de Pesquisa}

O problema de pesquisa é a inadequação das primitivas de PLS existentes para canais THz dominados por LoS e a necessidade de novos métodos que explorem as propriedades físicas desse domínio para geração de chaves secretas.

Considere um canal THz entre Alice e Bob com resposta em frequência $H_{AB}(f, t)$ influenciada pela absorção molecular $\alpha(f, d, \theta)$, onde $\theta = [T, \text{RH}, P]$ representa parâmetros atmosféricos (temperatura, umidade relativa, pressão). O problema central é:

Maximizar: $R_s = I(K_A; K_B) - I(K_A; K_E)$ \hfill (1)

sujeito a:

$H(K_A|K_B) \leq \epsilon_1$ (baixa discordância entre nós legítimos) \hfill (2)

$I(K_A; K_E) \leq \epsilon_2$ (informação vazada para Eve limitada) \hfill (3)

$\mathbb{E}[|\alpha(f, d, \theta)|] > \sigma_{\text{min}}$ (entropia mínima da absorção) \hfill (4)

onde $R_s$ é a taxa de sigilo (secrecy rate), $K_A$ e $K_B$ são as chaves extraídas por Alice e Bob, $K_E$ é a informação de Eve, $H(\cdot|\cdot)$ é entropia condicional, e $I(\cdot; \cdot)$ é informação mútua.

Para operacionalizar a hipótese, formulam-se as seguintes perguntas de pesquisa: \textbf{(PQ1)} Como as propriedades de absorção molecular e espalhamento do canal THz em ambientes (indoor e outdoor) podem ser modeladas e quantificadas para gerar chaves criptográficas com alta entropia e baixa correlação espacial? \textbf{(PQ2)} Qual metodologia e quais protocolos permitem a autenticação contínua de dispositivos e a geração eficiente de chaves em canais THz, garantindo segurança contra adversários passivos e apresentando trade-offs viáveis entre segurança, taxa de chave e complexidade computacional?

\subsection{Delimitação da Pesquisa}

Esta pesquisa focará na caracterização teórica e simulacional das primitivas de segurança, delimitando-se da seguinte forma: \textbf{Frequência:} análise concentrada na faixa de sub-THz (100-300 GHz), onde as janelas de transmissão atmosférica e os picos de absorção molecular são bem definidos e promissores para implementações iniciais do 6G. \textbf{Cenários:} investigação de cenários de comunicação indoor (industriais) e outdoor (urbanos), ambos estáticos e com baixa mobilidade. \textbf{Fora do Escopo:} implementação de hardware e prototipagem, bem como defesa contra ataques ativos complexos (e.g., jamming inteligente ou ataques de personificação por reconfiguração de superfícies), com foco em adversários passivos (interceptadores).
\section{REFERENCIAL TEÓRICO}
Esta pesquisa se baseia na transição da segurança computacional, tradicionalmente relegada às camadas superiores \cite{Alagic2020}, para a Segurança da Camada Física (PLS). A PLS explora as propriedades intrínsecas do canal sem fio, fundamentando-se na teoria de sigilo perfeito de Shannon \cite{Shannon1949} e no "canal de escuta" de Wyner \cite{Wyner1975}. O foco deste trabalho é a Geração de Chaves Secretas (SKG), que destila uma chave a partir da aleatoriedade do canal \cite{Maurer1993}, e a autenticação baseada em características físicas.

O ambiente de pesquisa é a faixa de Terahertz (THz) (0.1-10 THz), chave para o 6G, que oferece um canal de propagação singular \cite{Rappaport2019}. Este canal é dominado por alta perda de percurso e, crucialmente, por absorção molecular por gases atmosféricos \cite{Han2018, Gordon2022, ITU2019}. Essa dinâmica cria um canal ideal para PLS: é recíproco para concordância, mas variável e espacialmente único para gerar entropia e servir como uma impressão digital física.

A literatura, no entanto, apresenta duas lacunas centrais que esta tese visa preencher. Primeiro, as técnicas de SKG atuais dependem do multipercurso, que é escasso em enlaces THz LoS-dominantes \cite{Wang2017mmWave, Baracca2018}. Esta pesquisa propõe o uso da absorção molecular atmosférica como fonte primária de entropia, uma área que ainda carece de uma metodologia robusta \cite{Zhou2021}.

Segundo, embora o canal THz atue como uma Função Física Não Clonável (PUF) robusta contra spoofing devido à sua alta sensibilidade à localização \cite{Herder2014, Nguyen2022}, os modelos de autenticação são incipientes. A literatura carece de detectores que explorem as propriedades estatísticas distintivas da assinatura THz (esparsidade, sensibilidade atmosférica), uma lacuna que este trabalho abordará \cite{Basat2023}.
A relevância desta pesquisa se fundamenta em três dimensões. Primeiramente, a proliferação de dispositivos IoT, que deve ultrapassar 75 trilhões de unidades globalmente até 2030, exige soluções de segurança leves que não drenem recursos de bateria e processamento \cite{Wang2022}. A PLS atende a essa demanda, oferecendo segurança com baixo custo computacional.

Em segundo lugar, a PLS oferece uma defesa fundamental contra a ameaça da computação quântica. Enquanto a criptografia de chave pública atual será comprometida, a PLS é informacional-teoricamente segura. Esta pesquisa se posiciona não como um substituto, mas como um complemento à Criptografia Pós-Quântica (PQC) \cite{Alagic2020}: a PLS garante a distribuição segura de chaves de sessão (usadas por algoritmos simétricos, que são largamente resistentes a ataques quânticos), formando uma arquitetura de defesa em profundidade.

Cientificamente, este trabalho é original por focar na absorção molecular, uma característica do THz frequentemente tratada como um obstáculo, transformando-a em uma fonte de segurança. Isso preenche a lacuna deixada pelas técnicas de SKG baseadas em multipercurso, ineficazes no domínio THz \cite{Zhou2021}.

Estrategicamente, o desenvolvimento de primitivas de segurança para 6G é relevante para a competitividade tecnológica e soberania digital. As contribuições desta pesquisa podem influenciar futuras discussões de padronização em órgãos como o 3GPP e o ITU-R, além de ter impacto direto em aplicações que exigem alta confiabilidade, como saúde conectada (IoMT), automação industrial (IIoT) e comunicações táticas de defesa.

Reconhece-se que a PLS é vulnerável a ataques ativos (e.g., man-in-the-middle). No entanto, o foco desta pesquisa em autenticação física (PUF de canal) visa mitigar esses ataques de falsificação de identidade, fornecendo uma base segura sobre a qual a geração de chaves pode ocorrer.

\section{OBJETIVOS}

\subsection{Objetivo Geral}

Desenvolver e validar por meio de simulação computacional uma metodologia para caracterização e extração de primitivas de segurança física (geração de chaves e autenticação) baseadas nas propriedades únicas de propagação do canal no domínio Terahertz.

\subsection{Objetivos Específicos}

Os objetivos específicos são: (i) modelar o canal de propagação THz (100-300 GHz) em cenários indoor e outdoor, quantificando numericamente a entropia disponível da absorção molecular e a taxa de decorrelação espacial das assinaturas do canal; (ii) desenvolver e avaliar um protocolo de geração de chaves secretas (SKG) que explore a reciprocidade das medições de canal multi-banda, otimizando os processos de quantização e reconciliação para canais LoS-dominantes; (iii) propor e analisar um mecanismo de autenticação de dispositivos baseado no \textit{fingerprinting} da resposta de impulso do canal (CIR), estabelecendo limiares de decisão para detecção de ataques de \textit{spoofing} com alta probabilidade de detecção; (iv) analisar comparativamente os \textit{trade-offs} entre a taxa de geração de chaves, o nível de segurança (entropia da chave) e o \textit{overhead} de sinalização e processamento dos protocolos propostos.

\section{METODOLOGIA}

\subsection{Fundamentação Epistemológica e Tipo de Pesquisa}

Esta pesquisa adota um paradigma epistemológico \textbf{positivista}, buscando explicações causais e quantificáveis dos fenômenos físicos para a engenharia de segurança. O método de pesquisa será a \textbf{modelagem e simulação computacional}, justificado pela inexistência de plataformas de hardware 6G THz abertas e pelo alto custo e complexidade da prototipação experimental. A simulação permite a análise controlada e reprodutível de múltiplos cenários e parâmetros, embora suas conclusões dependam da fidelidade dos modelos físicos empregados.

\subsection{Modelagem do Canal THz}

A caracterização do canal será a base da tese.
\begin{alineas}
    \item \textbf{Absorção Molecular:} A atenuação atmosférica será modelada em alta resolução espectral (linha-a-linha) utilizando a base de dados \textbf{HITRAN 2020} \cite{Gordon2022} e os procedimentos da recomendação \textbf{ITU-R P.676-12} \cite{ITU2019}. Serão simulados parâmetros atmosféricos variáveis (T: 15-35°C, Umidade Relativa: 20-80\%, Pressão: 1 atm).
    \item \textbf{Multipercurso e Espalhamento:} Modelos de \textbf{ray-tracing} 3D serão implementados em MATLAB ou Python (usando bibliotecas como Sionna/Wireless Insite) para capturar as reflexões e espalhamentos difusos em cenários específicos.
    \item \textbf{Cenários:}
        \begin{enumerate}
            \item \textit{Indoor} (Industrial): Fábrica de $50 \times 50$ m, com obstáculos metálicos (maquinário).
            \item \textit{Outdoor} (Urbano): "Street canyon" com 100 m de comprimento, edifícios e baixa densidade de vegetação.
        \end{enumerate}
    \item \textbf{Validação do Modelo:} A entropia disponível será calculada (e.g., $\mathrm{H}(X)$) e a decorrelação espacial será validada comparando a função de autocorrelação $R(\Delta d)$ dos parâmetros do canal simulado com trabalhos teóricos \cite{Wang2017mmWave}.
\end{alineas}

\subsection{Desenho dos Protocolos de Segurança}

\begin{alineas}
    \item \textbf{Modelo de Adversário:} Será considerado um adversário \textbf{passivo} (Eve) que intercepta todas as transmissões, localizado a uma distância $d_E > \lambda/2$ dos nós legítimos.
    \item \textbf{Protocolo de Geração de Chaves (SKG):} O processo seguirá o fluxograma padrão (sondagem $\rightarrow$ quantização $\rightarrow$ reconciliação $\rightarrow$ amplificação).
        \begin{enumerate}
            \item \textit{Sondagem:} Medição da atenuação em $N$ sub-bandas THz (nas janelas e picos de absorção).
            \item \textit{Quantização:} Teste de quantizadores multi-nível adaptativos (baseados na média e variância do canal) versus quantização binária simples.
            \item \textit{Reconciliação:} Implementação do protocolo \textbf{Cascade} \cite{Brassard1993}, otimizando o número de passagens para o CIR esparso do THz.
            \item \textit{Amplificação:} Aplicação de funções hash universais (e.g., SHA-256) sobre os bits reconciliados.
        \end{enumerate}
    \item \textbf{Protocolo de Autenticação:}
        \begin{enumerate}
            \item \textit{Extração de Features:} A assinatura (PUF) será extraída do PDP, usando características como: energia recebida, atraso médio, \textit{RMS delay spread} e os tempos de chegada e amplitudes dos primeiros $K$ raios.
            \item \textit{Detecção:} Será usado um teste de hipótese (e.g., Neyman-Pearson) para decidir entre $H_0$ (usuário legítimo) e $H_1$ (adversário), com base na distância (e.g., Euclidiana ou de correlação) entre a assinatura medida e um \textit{template} armazenado.
        \end{enumerate}
\end{alineas}

\subsection{Plano de Simulação e Análise Estatística}

\begin{alineas}
    \item \textbf{Ferramentas:} MATLAB R2024a e/ou Python 3.10 (com bibliotecas SciPy, NumPy, Scikit-learn).
    \item \textbf{Execução:} Simulações de Monte Carlo com $10^5$ iterações por cenário para garantir significância estatística.
    \item \textbf{Cenários de Teste:} Variação de distância Alice-Bob (1m a 100m), distância de Eve, mobilidade (0 a 3 m/s), e número de sub-bandas de sondagem.
    \item \textbf{Métricas de Análise:}
        \begin{enumerate}
            \item SKG: Taxa de Geração de Chave (KGR) em bits/s, Taxa de Discordância de Bits (KDR) e Entropia da Chave (bits/símbolo).
            \item Autenticação: Probabilidade de Detecção (Pd) e Probabilidade de Falso Alarme (Pfa), gerando curvas ROC.
        \end{enumerate}
    \item \textbf{Análise Estatística:} Os resultados serão apresentados com intervalos de confiança de 95\%. Testes de significância (como ANOVA) serão usados para validar o impacto dos diferentes parâmetros (e.g., umidade vs. distância) nas métricas de desempenho.
\end{alineas}

\section{PLANO DE TRABALHO E CRONOGRAMA}

O plano de trabalho contempla: cursamento de 12 créditos em disciplinas (Comunicações Móveis - 4 créditos, Processos Estocásticos - 4 créditos, Processamento de Sinais - 4 créditos), com dedicação de 40 horas/semanais. O cronograma de execução (48 meses) está detalhado na Tabela~\ref{tab:cronograma_gantt}.

\begin{table}[!htbp]
    \centering
    \caption{Cronograma de Execução do Projeto (48 meses)}
    \label{tab:cronograma_gantt}
    \small
    \begin{tabular}{l c c c c c c c c}
        \toprule
        \textbf{Atividade} & \textbf{S1} & \textbf{S2} & \textbf{S3} & \textbf{S4} & \textbf{S5} & \textbf{S6} & \textbf{S7} & \textbf{S8} \\
        \midrule
        Revisão Bibliográfica e Projeto & X & - & - & - & - & - & - & - \\
        Fase I: Modelos de Canal THz & - & X & - & - & - & - & - & - \\
        Fase II: Protocolos (SKG e Autent.) & - & - & X & X & - & - & - & - \\
        Submissão Artigo (Conferência) & - & - & X & - & - & - & - & - \\
        Qualificação do Doutorado & - & - & - & X & - & - & - & - \\
        Submissão 1º Artigo (Periódico) & - & - & - & X & - & - & - & - \\
        Fases III/IV: Simulações e Análises & - & - & - & - & X & - & - & - \\
        Revisões do 1º Artigo & - & - & - & - & X & - & - & - \\
        Submissão 2º Artigo (Periódico) & - & - & - & - & - & X & - & - \\
        Redação da Tese & - & - & - & - & - & X & X & - \\
        Revisões do 2º Artigo & - & - & - & - & - & - & X & - \\
        Redação Final e Depósito & - & - & - & - & - & - & - & X \\
        Defesa Final & - & - & - & - & - & - & - & X \\
        \bottomrule
    \end{tabular}
\end{table}



% Bibliografia
\newpage
\bibliography{bibliografia}
\bibliographystyle{abntex2-num}

\end{document}
