A relevância desta pesquisa se fundamenta em três dimensões. Primeiramente, a proliferação de dispositivos IoT, que deve ultrapassar 75 trilhões de unidades globalmente até 2030, exige soluções de segurança leves que não drenem recursos de bateria e processamento \cite{Wang2022}. A PLS atende a essa demanda, oferecendo segurança com baixo custo computacional.

Em segundo lugar, a PLS oferece uma defesa fundamental contra a ameaça da computação quântica. Enquanto a criptografia de chave pública atual será comprometida, a PLS é informacional-teoricamente segura. Esta pesquisa se posiciona não como um substituto, mas como um complemento à Criptografia Pós-Quântica (PQC) \cite{Alagic2020}: a PLS garante a distribuição segura de chaves de sessão (usadas por algoritmos simétricos, que são largamente resistentes a ataques quânticos), formando uma arquitetura de defesa em profundidade.

Cientificamente, este trabalho é original por focar na absorção molecular, uma característica do THz frequentemente tratada como um obstáculo, transformando-a em uma fonte de segurança. Isso preenche a lacuna deixada pelas técnicas de SKG baseadas em multipercurso, ineficazes no domínio THz \cite{Zhou2021}.

Estrategicamente, o desenvolvimento de primitivas de segurança para 6G é relevante para a competitividade tecnológica e soberania digital. As contribuições desta pesquisa podem influenciar futuras discussões de padronização em órgãos como o 3GPP e o ITU-R, além de ter impacto direto em aplicações que exigem alta confiabilidade, como saúde conectada (IoMT), automação industrial (IIoT) e comunicações táticas de defesa.

Reconhece-se que a PLS é vulnerável a ataques ativos (e.g., man-in-the-middle). No entanto, o foco desta pesquisa em autenticação física (PUF de canal) visa mitigar esses ataques de falsificação de identidade, fornecendo uma base segura sobre a qual a geração de chaves pode ocorrer.
