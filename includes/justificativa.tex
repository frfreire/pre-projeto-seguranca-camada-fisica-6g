A relevância desta pesquisa se fundamenta em três dimensões. Primeiramente, a proliferação de dispositivos IoT, que deve ultrapassar 75 bilhões de unidades globalmente até 2030, exige soluções de segurança leves que não drenem recursos de bateria e processamento \cite{Wang2022}. A PLS atende a essa demanda, oferecendo segurança com overhead computacional mínimo.

Em segundo lugar, a PLS oferece segurança informacionalmente teórica baseada nas leis da física, independente de suposições sobre capacidade computacional de adversários. Diferentemente da criptografia tradicional que assume limitações computacionais (problemas NP-difíceis), a PLS garante sigilo explorando a vantagem física do canal legítimo sobre o canal do interceptador, conforme estabelecido por Wyner no conceito de "canal de escuta" \cite{Wyner1975}. Esta pesquisa se posiciona como camada de segurança adicional e complementar aos protocolos criptográficos convencionais, particularmente adequada para dispositivos IoT onde eficiência energética é crítica.

Cientificamente, este trabalho é original por focar na variabilidade temporal da absorção molecular como fonte de entropia para SKG. Revisão sistemática no IEEE Xplore (2019-2024) usando descritores "Terahertz", "Physical Layer Security" e "Key Generation" identificou 47 artigos. Destes, apenas Fang et al. \cite{Fang2022} abordam absorção molecular para segurança, mas exploram atenuação espacial estática para criar "zonas seguras" de comunicação de alcance limitado. Diferentemente, esta proposta utiliza a variabilidade temporal das condições atmosféricas como fonte de entropia compartilhada entre Alice e Bob. A reciprocidade do canal THz TDD (Time Division Duplex), combinada com sensibilidade à absorção molecular que varia com condições meteorológicas, cria oportunidade única para SKG que não foi explorada na literatura. Nenhum trabalho estabeleceu protocolo completo de SKG validado por simulação realística com modelos de ray-tracing e absorção linha-a-linha.

Análise preliminar de viabilidade indica que variações atmosféricas típicas (temperatura: $\pm$5-15°C, umidade relativa: $\pm$10-50\%, pressão: $\pm$5-15 mbar) causam mudanças de 0.4-1.5 dB/km na absorção THz \cite{ITU2019} em frequências próximas a linhas de H₂O. Para link de 10m em 300 GHz com sondagem de N=8 sub-bandas, estimativas preliminares sugerem entropia de 1.5-1.8 bits por medição por sub-banda, resultando em taxa de geração potencial de 12-14 bits/hora (outdoor) e 3-5 bits/hora (indoor controlado). Embora as taxas sejam baixas para streaming contínuo, são adequadas para geração de chaves de sessão (256-512 bits gerados em 1-2 dias), que é o caso de uso típico em redes IoT de longa duração.

Estrategicamente, o desenvolvimento de primitivas de segurança para 6G é relevante para a competitividade tecnológica e soberania digital. As contribuições desta pesquisa podem influenciar futuras discussões de padronização em órgãos como o 3GPP e o ITU-R, além de ter impacto direto em aplicações que exigem alta confiabilidade e segurança, como saúde conectada (IoMT), automação industrial (IIoT) e comunicações táticas de defesa. Em fábricas inteligentes, por exemplo, a autenticação por PUF de canal THz pode prevenir ataques de personificação em AGVs (Automated Guided Vehicles) operando em células de manufatura com requisitos de comunicação ultra-confiável e baixa latência.

Reconhece-se que a PLS é vulnerável a ataques ativos (e.g., man-in-the-middle). No entanto, o foco desta pesquisa em autenticação física (PUF de canal) visa mitigar esses ataques de falsificação de identidade, fornecendo uma base segura sobre a qual a geração de chaves pode ocorrer. A alta sensibilidade espacial do canal THz (decorrelação em distâncias inferiores a $\lambda$/2 $\approx$ 0.5 mm a 300 GHz) tornaria extremamente difícil para um adversário replicar a assinatura exata do canal sem estar fisicamente co-localizado com o usuário legítimo.