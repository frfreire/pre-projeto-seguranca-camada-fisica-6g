\section{METODOLOGIA}

\subsection{Fundamentação Epistemológica e Tipo de Pesquisa}

Esta pesquisa adota um paradigma epistemológico positivista, buscando explicações causais e quantificáveis dos fenômenos físicos para a engenharia de segurança. O método será a modelagem e simulação computacional, justificado pela inexistência de plataformas de hardware 6G THz abertas e pelo alto custo da prototipação experimental. A simulação permite análise controlada e reprodutível de múltiplos cenários, embora suas conclusões dependam da fidelidade dos modelos físicos empregados.

\subsection{Modelagem do Canal THz}

A caracterização do canal será a base da tese. A atenuação atmosférica por absorção molecular será modelada em alta resolução espectral (linha-a-linha) utilizando a base HITRAN 2020 \cite{Gordon2022} e os procedimentos da recomendação ITU-R P.676-12 \cite{ITU2019}, simulando parâmetros atmosféricos variáveis (T: 15-35°C, Umidade Relativa: 20-80\%, Pressão: 1 atm). Modelos de ray-tracing 3D serão implementados em Python (bibliotecas Sionna/Wireless Insite) para capturar reflexões e espalhamentos difusos em cenários específicos: (i) indoor industrial (fábrica 50×50m com obstáculos metálicos) e (ii) outdoor urbano (street canyon 100m com edifícios). A validação incluirá cálculo da entropia disponível H(X) e comparação da função de autocorrelação R(Δd) com trabalhos teóricos \cite{Wang2017mmWave}.

\subsection{Protocolos de Segurança e Plano de Simulação}

Será considerado um adversário passivo (Eve) localizado a distância $d_E > \lambda/2$ dos nós legítimos. O protocolo de geração de chaves (SKG) seguirá o fluxograma padrão: sondagem de atenuação em N sub-bandas THz (janelas e picos de absorção), quantização multi-nível adaptativa (baseada em média e variância do canal) versus binária simples, reconciliação via protocolo Cascade \cite{Brassard1993} otimizado para CIR esparso, e amplificação com funções hash universais (SHA-256). Para autenticação, a assinatura (PUF) será extraída do PDP usando características como energia recebida, atraso médio, RMS delay spread e tempos/amplitudes dos primeiros K raios, com teste de hipótese Neyman-Pearson para decidir entre usuário legítimo e adversário. As simulações serão executadas em Python 3.10 (SciPy, NumPy, Scikit-learn) via Monte Carlo com $10^5$ iterações por cenário, variando distância Alice-Bob (1-100m), distância de Eve, mobilidade (0-3 m/s) e número de sub-bandas. As métricas incluem: para SKG, taxa de geração de chave (KGR), taxa de discordância de bits (KDR) e entropia da chave; para autenticação, probabilidade de detecção (Pd) e falso alarme (Pfa) com curvas ROC. Os resultados serão apresentados com intervalos de confiança 95\% e testes ANOVA para validar o impacto dos parâmetros nas métricas de desempenho.
