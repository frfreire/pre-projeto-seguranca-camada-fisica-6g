\section{REFERENCIAL TEÓRICO}
Esta pesquisa se baseia na transição da segurança computacional, tradicionalmente relegada às camadas superiores \cite{Alagic2020}, para a Segurança da Camada Física (PLS). A PLS explora as propriedades intrínsecas do canal sem fio, fundamentando-se na teoria de sigilo perfeito de Shannon \cite{Shannon1949} e no "canal de escuta" de Wyner \cite{Wyner1975}. O foco deste trabalho é a Geração de Chaves Secretas (SKG), que destila uma chave a partir da aleatoriedade do canal \cite{Maurer1993}, e a autenticação baseada em características físicas.

O ambiente de pesquisa é a faixa de Terahertz (THz) (0.1-10 THz), chave para o 6G, que oferece um canal de propagação singular \cite{Rappaport2019}. Este canal é dominado por alta perda de percurso e, crucialmente, por absorção molecular por gases atmosféricos \cite{Han2018, Gordon2022, ITU2019}. Essa dinâmica cria um canal ideal para PLS: é recíproco para concordância, mas variável e espacialmente único para gerar entropia e servir como uma impressão digital física.

A literatura, no entanto, apresenta duas lacunas centrais que esta tese visa preencher. Primeiro, as técnicas de SKG atuais dependem do multipercurso, que é escasso em enlaces THz LoS-dominantes \cite{Wang2017mmWave, Baracca2018}. Esta pesquisa propõe o uso da absorção molecular atmosférica como fonte primária de entropia, uma área que ainda carece de uma metodologia robusta \cite{Zhou2021}.

Segundo, embora o canal THz atue como uma Função Física Não Clonável (PUF) robusta contra spoofing devido à sua alta sensibilidade à localização \cite{Herder2014, Nguyen2022}, os modelos de autenticação são incipientes. A literatura carece de detectores que explorem as propriedades estatísticas distintivas da assinatura THz (esparsidade, sensibilidade atmosférica), uma lacuna que este trabalho abordará \cite{Basat2023}.